\section{Einleitung}

In diesem Versuch soll ein Motorkühlungsüberwachungsautomat simuliert werden. Ein Motor wird über drei Pumpen mit Kühlmittel versorgt. An jeder Pumpe ist ein Strömungssensor montiert, der eine logische 1 ausgibt, wenn die Pumpe funktioniert. Die drei Signale werden von einem Zustandsautomaten ausgewertet, welcher wiederum drei LEDs (Rot, Gelb, Grün) ein- oder ausschaltet, um den Zustand der Pumpen zu signalisieren.

\subsection{Verwendete Software}

Für den Versuch wird folgende Software verwendet:

\begin{table}[ht]
    \centering
    \begin{tabular}{|c|c|c|c|}\hline
    \tbf{Gerätetyp}             & \tbf{Bezeichnung}             \\ \hline
    Betrebsystem                & Windows 10 Pro                \\ \hline
    HDL-Simulationsumgebung     & ModelSim PE Student Edition   \\ \hline
    \end{tabular}
    \caption{Auflistung der Software}
\end{table}
